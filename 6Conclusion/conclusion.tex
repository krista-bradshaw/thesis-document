\chapter{Conclusion}
\section{Summary and Conclusions}
This thesis explored the impact query variations have on the robustness of retrieval pipelines. The overarching objective drove the research to understand how query variations influence the retrieval effectiveness of diverse information retrieval systems. The study generated valuable insights and outcomes that collectively address the core research questions by extensively exploring various datasets, models, and query variations. Here, a synthesis of what has been accomplished, the pivotal conclusions drawn from the results, and reflections on potential avenues for future research are provided.

In this study, two primary research questions guided the investigation—the first research question aimed to reproduce an original study's experiments, methodologies, and findings. The primary focus was to investigate the robustness of retrieval pipelines to query variations, explicitly emphasising the effect of query variations on different ranking models. The results largely align with the original study's findings, reaffirming the lack of robustness of retrieval pipelines to query variations. Overall, the successful reproduction of the original study's results enhances the credibility of the conclusions drawn regarding the impact of query variations on retrieval pipelines and provides a strong foundation for further exploration in this critical area of information retrieval research.

The second research question, regarding expansion, delved into the broader perspective on the original study's conclusions by repeating the work on a new and unexplored dataset, DL-TYPO. The outcomes not only upheld the original findings but also revealed the dataset's unique characteristics, accentuating the need for careful consideration of dataset-specific attributes when assessing the impact of query variations. With compelling consistency across multiple datasets and models, the original study's findings were validated, emphasising the integral role of query variations in evaluating retrieval systems. To gain a nuanced understanding of the influence of different query variation categories, variations were systematically categorised into misspellings, paraphrasing, naturality, and word order, uncovering that misspelling variations consistently had the most detrimental impact on retrieval effectiveness, further corroborating the significance of the original study's insights.

Additionally, the robustness of different model categories was explored, showing that models within the same category exhibit similar behaviour when confronted with query variations. A preference among transformer-based language models for natural language queries was identified in this context. In contrast, word order had a minimal impact on these models, aligning with recent research trends.

To enhance the retrieval outcomes influenced by query variations, rank fusion techniques were experimented with, highlighting the potential for query variation methods to augment retrieval effectiveness significantly. The results were consistent with the original study. Combining query variations using RRF generally mitigated the decreases in retrieval effectiveness observed when using query variations individually. However, it was noteworthy that RRF did not consistently outperform using the original query. This reaffirms that while rank fusion techniques can improve retrieval performance under certain circumstances, they need to prove to be a solution for the challenges posed by query variations.

\section{Possible Future Work}
The outcomes of this research underscore the significant influence of query variations on the robustness of retrieval pipelines, shedding light on the intricacies that govern these variations in various datasets, models, and variation category contexts. This opens up new avenues for future exploration in information retrieval. Several promising directions for further research are outlined below:

\begin{itemize}
    \item Optimising Query Variation Methods: The investigation of advanced techniques for generating query variations tailored to specific information retrieval systems represents a promising research path. Such optimisation may help mitigate the potentially adverse impact of specific variations on retrieval effectiveness.
    \item Leveraging Neural Models: There is room for further exploration into the effectiveness of neural models in handling query variations. Developing more robust deep learning architectures specialised for this task could enhance the adaptability of retrieval systems to varying queries.
    \item Real-world Applications: Extending this research to examine the practical implications of query variations in real-world information retrieval systems and user experiences is essential. This could involve evaluating the impact of query variations on user satisfaction, interaction, and engagement with retrieval platforms.
\end{itemize}

These research directions hold the potential to advance our understanding of query variations and their implications in the context of information retrieval, contributing to more effective and user-centric retrieval systems.

In conclusion, this study adds significant insights to the information retrieval domain, emphasising the importance of query variations and their substantial impact on retrieval pipeline robustness. As the field evolves, these findings will guide researchers, system developers, and practitioners in crafting more effective and adaptive information retrieval systems. The journey of understanding query variations has only just begun, and the path forward promises further exploration and discovery in the ever-evolving landscape of information retrieval.