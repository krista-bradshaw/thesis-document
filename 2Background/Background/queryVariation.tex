\section{Query Variations}
Query variation is a fundamental concept in IR, encompassing a wide array of expressions and formulations that users employ when seeking specific information or conducting online searches \cite{zendel}. These variations can take shape in numerous ways, including alterations in word order, sentence structure, and shifts in phrasing. To illustrate, let's delve into the realm of baking and consider two queries: "How to bake a cake from scratch?" and "Homemade cake recipe?" Although these two queries exhibit nuanced differences, they essentially convey the same information need, showcasing the diverse ways in which users can articulate their search intents.

Query variations span a broad spectrum, encompassing distinct categories, each with unique characteristics and challenges. Some common categories of query variations include specialisation, aspect change, misspelling, naturality, and paraphrasing \cite{penha2022}. Specialisation involves refining a general query into a more specific one; aspect change entails altering the focus of the query while retaining the core intent; misspelling can introduce errors or typos; natural language transformation might involve changing from a question to a statement and paraphrasing entails expressing the same information need in different words.

Addressing query variation is of paramount importance in IR because it has a profound impact on the performance and effectiveness of search systems. Recognising and accommodating the diverse ways users express their information needs is pivotal for enhancing the capacity of IR systems to retrieve relevant documents. Several techniques are employed to manage query variation effectively in IR systems, including query suggestion, reformulation, and expansion:
\begin{itemize}
    \item Query Expansion: Query expansion is a technique used to broaden the scope of the original query. This is achieved by adding synonyms, related terms, or additional keywords to the initial query. The goal is to retrieve a more comprehensive set of relevant documents and improve the recall of the search results.
    \item Query Reformulation: Query reformulation involves transforming the initial query into a new one that better aligns with the user's information needs. This can include changing the wording, structure, or focus of the query to increase its precision and relevance to the user.
\end{itemize}

By implementing these techniques and considering the multifaceted nature of query variations, IR systems can provide users with an enhanced and more tailored search experience. This not only improves the quality of search results but also ensures that users can find the information they need, even if their queries vary in expression and formulation.