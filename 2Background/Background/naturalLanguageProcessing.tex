\section{Natural Language Processing}
\subsection{Ranking}
\note{add a diagram}
Ranking is a pivotal component within information retrieval systems, central to organising a collection of textual documents or search results based on their relevance to a given query or topic. The ranking process in these algorithms involves utilising multiple features to assign numerical scores to individual documents. These scores subsequently serve as the basis for arranging the documents in descending order, ensuring that the most pertinent ones appear at the forefront of the list \cite{wu}.

The significance of ranking transcends many applications, impacting the functionality of diverse systems, including search engines, chatbots, and question-answering platforms. To comprehend the ranking process, it is essential to consider several key aspects:

\begin{itemize}
    \item Feature Assessment: Ranking algorithms employ sophisticated techniques to assess and quantify the relevance of documents. These techniques encompass many factors, including keyword matching, semantic analysis, user behaviour patterns, and contextual cues.
    \item User-Centric Adaptation: Ranking algorithms operate dynamically, allowing information retrieval systems to adapt to evolving user queries and content. This adaptability ensures that users receive increasingly accurate and context-aware results.
    \item Enhanced User Experience: The impact of ranking extends to user experience and system performance. By prioritising the presentation of the most relevant content, ranking algorithms enhance user satisfaction and the overall efficiency of information retrieval processes.
    \item Ongoing Advancements: The ranking field continually evolves, with the integration of advanced machine learning and natural language processing methods contributing to the refinement of information retrieval systems. These advancements enhance their effectiveness and applicability across a broad spectrum of domains.
\end{itemize}

\subsection{Rank Fusion}
\note{add my references - maybe add a diagram or dot points}
In information retrieval, rank fusion is a valuable technique that amalgamates ranked document lists generated by multiple retrieval methods or models. The primary objective is to enhance the retrieval process's overall efficacy by integrating outcomes from various sources.

The fundamental concept underlying rank fusion revolves around assigning scores or weights to individual documents within the ranked lists acquired from diverse retrieval methods. These set values are subsequently employed in constructing a novel merged list where documents are arranged based on their rankings. A commonly utilised approach for rank fusion is Reciprocal Rank Fusion (RRF).

RRF is a technique used in information retrieval to combine the rankings generated by multiple retrieval methods or queries. It's advantageous when you have different sources of ranking scores, such as various retrieval models or query variations, and you want to aggregate them to improve overall retrieval performance. The core idea behind RRF is to assign a reciprocal rank score to each document in the individual rankings. Reciprocal rank is a measure that gives higher scores to documents that appear higher in the rankings. Then, these reciprocal rank scores from different rankings are summed up for each document, and the documents are re-ranked based on this aggregated score.

Rank fusion is advantageous when distinct retrieval techniques possess complementary strengths and weaknesses. Combining their findings makes it feasible to bolster overall performance in retrieving relevant documents while potentially enhancing quality standards. Nevertheless, meticulous design and evaluation of rank fusion methodologies remain crucial to effectively capture each source's advantages and generate meaningful merged rankings accordingly.

\subsection{Training}
Training denotes the process of instructing a machine learning model to perform specific linguistic tasks with a high degree of accuracy. These tasks encompass a wide range, including but not limited to text classification, sentiment analysis, and machine translation, and are pivotal to the broader field of NLP \cite{bert}. During the training, a substantial corpus of text data is input into the model, paired with corresponding target labels that convey the desired output for each input example. Through iterative exposure to this labelled data, the model endeavours to discern intricate relationships and patterns that underlie the language-based tasks at hand \cite{nlp}.

The training phase is characterised by recurrent data processing, often called epochs. Throughout these iterations, the model's internal parameters are fine-tuned to progressively enhance its capacity to make accurate predictions and generate meaningful linguistic representations. This fine-tuning is facilitated by optimisation techniques that adjust the model's parameters to optimise its overall performance. The ultimate objective of the training process is to cultivate models that exhibit high accuracy in their predictions and operate with computational efficiency. These well-trained models are then deployed across various NLP applications, where they can efficiently process natural language text, offering valuable insights and enabling a myriad of language-related tasks to be automated and executed effectively.

Additionally, consider adding a diagram that visually represents the ranking and rank fusion processes to enhance the clarity and understanding of these concepts.