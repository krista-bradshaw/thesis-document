\chapter{Abstract}
This thesis investigates the significance of query variations in information retrieval pipelines, aiming to replicate and expand upon the seminal research conducted by Gustavo Penha, Arthur Câmara, and Claudia Hauf. The central objective of this study is to examine how using query variations, which are subtly modified versions of search queries, can exert a discernible influence on retrieval efficiency.

The research methodology adopted in this study involved the application of ten established methods for automatically generating query variations. These query variations were systematically introduced into three datasets: ANTIQUE, TREC-DL-2019, and DL-TYPO. This rigorous process revealed intriguing findings, notably that methods involving synonyms and paraphrasing often resulted in queries with nuanced meanings, primarily due to the compounded effect of misspellings within the queries.

The experimental design utilised BM25 as an initial retrieval model, followed by the re-ranking of the top 100 retrieved results using various models: RM3, KNRM, CKNRM, BERT, EPIC, and T5. This procedure was executed for both the original search queries and their respective variations, with the evaluation of retrieval effectiveness being quantified through the nDCG@10 metric. Additionally, the study examined the impact of combining rankings from different query variation categories, providing a comprehensive assessment of the potential benefits of cross-category query combinations.

The outcomes of the experiments showcased that query variations indeed possess a profound influence on retrieval effectiveness. The replication phase of the study effectively reaffirmed the importance of investigating query variations, which was further confirmed when introducing DL-TYPO. This phase also underscored the necessity of considering dataset-specific characteristics when assessing the impact of query variations. It was found that 51, 41, and 32 instances in ANTIQUE, TREC-DL-2019, and DL-TYPO, respectively, exhibited statistically significant declines in retrieval effectiveness when query variations were introduced. Although combining query variation categories often yielded improvements in results, it generally fell short of matching the effectiveness of the original, unaltered queries.

In summation, this thesis shows the profound influence that query variations have on the robustness of retrieval pipelines, reaffirming their pivotal role in evaluating retrieval systems. The study successfully replicated and expanded upon the original findings, offering valuable insights into the impact of different query variation generators. The code and associated materials are accessible via the following link: \url{https://github.com/krista-bradshaw/query_variation_generators}.