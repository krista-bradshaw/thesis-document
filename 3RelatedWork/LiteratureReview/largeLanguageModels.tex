\section{Information Retrieval}
Recent studies in Information Retrieval have been driven by a growing interest in enhancing the performance and efficiency of Large Language Models (LLMs) in Natural Language Processing (NLP) tasks, particularly text generation and understanding. These studies leverage pre-trained models like GPT \cite{gpt}, and BERT \cite{bert}, which have consistently achieved state-of-the-art results across a range of NLP tasks. Pre-trained models are a foundation for various NLP applications, facilitating their adaptability and effectiveness.

One noteworthy trend in this domain is exploring transfer learning techniques, such as fine-tuning \cite{bert} and domain adaptation \cite{wu}. Fine-tuning allows practitioners to customise pre-trained LLMs for specific tasks or domains, while domain adaptation helps models perform optimally in specialised contexts. These techniques are critical contributors to the versatility of LLMs in addressing real-world challenges.

Looking ahead, the research landscape is poised to delve into more complex and diverse NLP tasks, extending beyond traditional text generation and classification. Researchers aim to develop innovative methods to enhance the efficiency and scalability of LLMs further, making them applicable to an even more comprehensive array of domains and applications \cite{bailey, zuccon, benham}.

\subsection{Improvement of Retrieval Pipelines}
The recent literature surrounding retrieval pipelines has been marked by a concerted effort to elevate the effectiveness and efficiency of handling large-scale datasets and accommodating diverse query types. Researchers have embraced deep learning techniques, such as deep reinforcement learning and neural networks, to optimise the search process \cite{penha2022, bailey, gao}. These techniques allow for development of more sophisticated retrieval systems that can adapt to different contexts and user needs.

Another significant area of investigation pertains to query expansion and query prediction techniques, which are crucial in improving the accuracy and relevance of search results \cite{zendel, lu, moffat}. By enhancing query understanding and prediction, retrieval systems become more adept at delivering meaningful and targeted search results to users.

The development of novel evaluation metrics, including diversity measures, represents a significant stride forward in assessing the quality of search results. These metrics go beyond traditional accuracy-based measures and focus on determining the relevance, novelty, and richness of retrieved information. This holistic approach to evaluation, as seen in Sokolova et al.'s work \cite{sokolova}, is essential for ensuring that retrieval pipelines truly enhance user satisfaction and the overall search experience.

While numerous studies are dedicated to improving retrieval pipelines, there is a concurrent focus on identifying and addressing vulnerabilities and weaknesses in these systems. Gao et al.'s work \cite{gao}, which introduces a framework designed to generate misleading queries, highlights the importance of robustness and security in retrieval systems. Identifying and mitigating potential weaknesses is essential for developing reliable and trustworthy retrieval pipelines.

\subsection{Evaluation}
Recent literature on evaluating Large Language Models (LLMs) emphasises the need for more diverse and robust evaluation methods to gauge their performance across various NLP tasks accurately. Traditional metrics like accuracy, which assess a model's ability to classify text correctly, have been scrutinised for their applicability in real-world scenarios \cite{sokolova, ribeiro}.

Moffat et al.'s introduction of the C/W/L framework \cite{moffat} has presented a novel approach to evaluation that takes into consideration user behaviour and engagement. This framework factors the likelihood of a user continuing to the following document while exploring search results, providing a more nuanced and user-centric evaluation of LLMs' performance. This approach aims to capture the "expected goal of search," ultimately leading to more accurate predictions of user intent and needs.

Ribeiro et al.'s proposed evaluation model \cite{ribeiro} offers an innovative perspective by addressing the limitations of conventional evaluation measures and methods. Their experiments have uncovered previously unnoticed shortcomings in NLP models that had undergone extensive testing. This highlights the importance of continually evolving evaluation methodologies to ensure that LLMs are rigorously assessed and refined for real-world applications.