\begin{table}
\caption{Outline of each of the ten variation generator methods used \cite{penha2022}.}
\label{tab:var-gens}
\begin{tabularx}{\columnwidth}{l|l|X}
\textbf{Category} &
  \textbf{Method} &
  \textbf{Definition} \\ \hline
\multirow{3}{*}{Misspelling} &
  NeighbCharSwap &
  Swaps two neighbouring characters from a random query term (excluding stopwords). \\ \cline{2-3} 
 &
  RandomCharSub &
  Replaces a random character from a random query term (excluding stopwords) with a randomly chosen new ASCII character. \\ \cline{2-3} 
 &
  QWERTYCharSub &
  Replaces a random character of a random query term (excluding stopwords) with another character from the QWERTY keyboard, mimicking typing errors. \\ \hline
\multirow{2}{*}{Naturality} &
  RemoveStopWords &
  Removes all stopwords from the query, transforming natural language queries into keyword queries. \\ \cline{2-3} 
 &
  T5DescToTitle &
  Applies an encoder-decoder transformer model (T5) fine-tuned to generate the title of a TREC topic title based on the topic description. \\ \hline
Ordering &
  RandomOrderSwap &
  Randomly swaps two words of the query, shuffling word order. \\ \hline
\multirow{4}{*}{Paraphrasing} &
  BackTranslation &
  Applies a translation method to the query, translating it to a pivot language and back to the original language, generating paraphrases. \\ \cline{2-3} 
 &
  T5QQ &
  Utilises an encoder-decoder transformer model (T5) fine-tuned to generate a paraphrase question from the original question \cite{desctitle}. \\ \cline{2-3} 
 &
  WordEmbedSynSwap &
  Replaces a non-stop word by a synonym based on the nearest neighbour word in the embedding space, using counter-fitted Glove embeddings \cite{glovefit}. \\ \cline{2-3} 
 &
  WordNetSynSwap &
  Replaces a non-stop word by the first synonym found on WordNet (https://wordnet.princeton.edu/); if no valid synonyms are available, it does not output a valid variation. \\ \hline
\end{tabularx}
\end{table}